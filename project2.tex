\documentclass[10pt,a4paper]{article}
\usepackage[latin1]{inputenc}
\usepackage{amsmath}
\usepackage{amsfonts}
\usepackage{amssymb}
\usepackage{mathtools}
\usepackage{bm}
\usepackage{standalone}
% Use \bm{x} for vectors/matrices in bold AND italic

\newcommand{\vectornorm}[1]{\left\|#1\right\|}

\begin{document}
\title{Programming project in TMA4220 \\ Part 2B \\ Helmholtz' equation for electromagnetic waves}
\author{Anders Maximilian Opskar Voldsund \\ Andreas Sebastian Borgen Longva \\ Espen August Johansen Velsvik}
\maketitle


\section*{Introduction}
The Helmholtz equation,
\begin{equation}
\begin{aligned}
\label{Helmholtz}
\Delta \psi + k^2 \psi = g,
\end{aligned}
\end{equation}
is a time-independent form of the wave equations which often arise in physical problems involving the computation of electromagnetic radiation. In the modeling of electromagnetic waves $\psi$ represents the Fourier transform of either the electric field $E$ or the magnetic field $H$. $k$ is called the wavenumber, and from the derivation of the Helmholtz' equation from Maxwell's equations in vacuum it is seen that $k = \frac{2\pi f}{c}$. Here $f$ is the frequency of the electromagnetic waves and $c = \frac{1}{\sqrt{\mu_0 \epsilon_0}}$ is the speed of light, where $\mu_0$ and $\epsilon_0$ are the permeability and permittivity in vacuum.

\section*{The Weak Formulation}
We consider the following Dirichlet problem:
\begin{equation}
\begin{aligned}
\label{Helmholtz_Dir}
\Delta \psi + k^2 \psi &= g  \quad \text{in} \, \Omega \\
\psi &= 0 \quad \text{on} \, \partial \Omega.
\end{aligned}
\end{equation}
To derive a weak formulation the equation is multiplied with a suiting test function $v \in X $ and integrate over the domain $\Omega$. The left hand side gives:
\begin{align*}
\int_\Omega \nabla \psi &+ k^2 \psi v \, \mathrm{d} \Omega \\
&\Downarrow \\
\int_{\partial \Omega} \frac{\partial \psi}{\partial n} v \, \mathrm{d} \Omega + \int_\Omega & k^2 \psi v - \nabla \psi \nabla v \, \mathrm{d} \Omega.
\end{align*}
At this point we can make a choice for the space of $v$. We choose $v$ to lie in the following space:
\begin{align*}
X = H^1_0(\Omega) = \left\{ v \in H^1(\Omega) : v = 0 \text{ on }   \partial \Omega \right\},
\end{align*}
where
\begin{align*}
H^1(\Omega) = \left\{ v: \Omega \rightarrow \mathbb{R}: v \in L^2(\Omega), v' \in L^2(\Omega) \right\}.
\end{align*}
Since $v$ disappears on the boundary we are left with the following expression:
\begin{align*}
\int_\Omega & k^2 \psi v - \nabla \psi \nabla v \, \mathrm{d} \Omega = \int_\Omega gv \, \mathrm{d} \Omega.
\end{align*}
Let now $a(u,v)$ and $F(v)$ be defined as the following functionals:
\begin{align*}
a(u,v) &= \int_\Omega  \nabla u \nabla v - k^2 u v  \, \mathrm{d} \Omega  \quad \text{and} \\
F(v) &= - \int_\Omega gv \, \mathrm{d} \Omega.
\end{align*}
This implies that $a(u,v) = F(v)$. The following problem is called the weak formulation of (\ref{Helmholtz_Dir}):
\begin{equation}
\begin{aligned}
\label{Helmholtz_weak}
\text{find} \, u \in H^1_0(\Omega): \quad a(u,v) = F(v) \quad \forall \, v \in H^1_0(\Omega).
\end{aligned}
\end{equation}
The functional $a(u,v)$ is continuous. This is seen by the following:
\begin{align*}
|a(u,v)| &= |\int_\Omega  \nabla u \nabla v - k^2 u v \, \mathrm{d} \Omega| \\
 		& \leq |\int_\Omega \nabla u \nabla v \, \mathrm{d} \Omega| + k^2 |\int_\Omega u v \, \mathrm{d} \Omega| \\
 		& \leq \vectornorm{\nabla u}_{L^2(\Omega)} \vectornorm{\nabla v}_{L^2(\Omega)} + k^2 \vectornorm{u}_{L^2(\Omega)} \vectornorm{v}_{L^2(\Omega)} \\
 		& \leq C \vectornorm{u}_{L^2(\Omega)} \vectornorm{v}_{L^2(\Omega)} + k^2 \vectornorm{u}_{L^2(\Omega)} \vectornorm{v}_{L^2(\Omega)} \\
 		& \leq C \vectornorm{u}_{H^1(\Omega)} \vectornorm{v}_{H^1(\Omega)} + k^2 \vectornorm{u}_{H^1(\Omega)} \vectornorm{v}_{H^1(\Omega)} \\
 		& = M \vectornorm{u}_{H^1(\Omega)} \vectornorm{v}_{H^1(\Omega)}.
\end{align*}
The functional $a(u,v)$ is also coercive under certain conditions, since
\begin{align*}
a(u,u) &= \int_\Omega  \nabla u \nabla u - k^2 u u \, \mathrm{d} \Omega \\
 		& = \int_\Omega  |\nabla u|^2 \, \mathrm{d} \Omega - \int_{\Omega} k^2 |u|^2 \, \mathrm{d} \Omega \\
 		& =  \vectornorm{\nabla u}_{L^2(\Omega)}^2 - k^2 \vectornorm{u}_{L^2(\Omega)}^2\\
 		& = \vectornorm{u}_{H^1(\Omega)}^2 - \vectornorm{u}_{L^2(\Omega)}^2 - k^2 \vectornorm{u}_{L^2(\Omega)}^2\\
 		& = \vectornorm{u}_{H^1(\Omega)}^2 - C \vectornorm{u}_{L^2(\Omega)}^2\\
 		& \geq \vectornorm{u}_{H^1(\Omega)}^2 - C \vectornorm{u}_{H^1(\Omega)}^2\\
 		& = \alpha \vectornorm{u}_{H^1(\Omega)}^2
\end{align*}
Lastly the functional is bilinear, since it is linear in each component. The bilinear functional $F(v)$ is continuous:
\begin{align*}
|F(v)| &= |\int_\Omega gv \, \mathrm{d} \Omega| \leq \vectornorm{g}_{L^2(\Omega)} \vectornorm{v}_{L^2(\Omega)} \\
       &= M \vectornorm{v}_{L^2(\Omega)},
\end{align*}
and from the Lax-Milgram theorem there exists a unique solution to (\ref{Helmholtz_weak}).

\section*{Relation to The Wave Equation}
As already stated, the Helmholtz equation for the modeling of electromagnetic waves can be derived from the Maxwell's equations in vacuum
\begin{align*}
\nabla \times E &= - \frac{\partial H}{\partial t} \\
\nabla \times H &= \mu_0 \epsilon _0 \frac{\partial E}{\partial t} \\
\nabla \cdot E &= 0 \\
\nabla \cdot H &= 0
\end{align*}

where $E: \Omega \times \mathbb{R}_+ \rightarrow \mathbb{R}^3$ is the electric field and $H: \Omega \times \mathbb{R}_+ \rightarrow \mathbb{R}^3$ is the magnetic field. It can be shown (\cite{Project}) that the equations can be decoupled into the wave equation for each component of the electric and magnetic field,
\begin{equation}
\begin{aligned}
\label{Wave}
\frac{\partial^2 \phi}{\partial t^2} - c^2 \Delta \phi = 0, 
\end{aligned}
\end{equation}
where $\phi$ is a spatial component for $E$ or $H$. To arrive at the Helmholtz equation we take the Fourier transform of $\phi$. From \cite{Helmholtz-Wave} we have that if $u(x,t) = \psi (x)e^{-i \omega t}$ satisfies (\ref{Wave}), then $\psi (x)$ is a solution to (\ref{Helmholtz}). This means that by numerically solving the Helmholtz equation, we are modeling the amplitude of the wave for one of the spatial components of the two fields $E$ and $H$.


\section*{Implementation of Helmholtz equation using FEM}


\section*{Mesh Generation}
Andreas needs to write about this!!!!!!!!!!!!

\section*{Unit box, Dirichlet, 1 point source}
One source in the middle of a unit box


\section*{Unit box, Dirichlet/Robin, several point sources }


\section*{Looking at increasing $\omega$ and mesh size}

\section*{Introducing Spatially Varying Wave Number}
By introducing different materials in our domain we are able to model a more realistic scenario.  assigning permeability $\mu$ and permittivity $\epsilon$ to  

\section*{Attenuation Of The Signal}


\section*{Final result: Modeling A House, varying materials, 1 or several point sources }



\documentclass[10pt,a4paper]{article}
\usepackage[latin1]{inputenc}
\usepackage{amsmath}
\usepackage{amsfonts}
\usepackage{amssymb}
\begin{document}


\section*{Robin-Type Boundary Conditions}
If we instead of the Dirichlet boundary conditions take a look at a Robin-type boundary condition, given by
\begin{align}\label{eq:robinType}
\frac{\partial \psi}{\partial n} - ik\psi = 0,
\end{align}
the idea is that we will obtain a solution that ensures that all waves will leave $\Omega$ and none will enter $\Omega$. \cite{Project}

Starting with the Helmholtz equation, we multiply by a test function $v$ and integrate over the domain $\Omega$. 

\begin{align}
\int_\Omega \Delta \psi v \, d\Omega + \int_\Omega k^2 \psi v \, d\Omega = \int_\Omega gv \, d\Omega
\end{align}

Doing integration by parts on the first term leads to
\begin{align}\label{eq:robinIntParts}
\int_{\partial \Omega} \frac{\partial \psi}{\partial n} v \, d\gamma - \int_\Omega \nabla \psi \nabla v \, d\Omega + \int_\Omega k^2 \psi v \, d\Omega = \int_\Omega gv \, d\Omega
\end{align}

Inserting \eqref{eq:robinType} into \eqref{eq:robinIntParts} we get our weak solution
\begin{align}\label{eq:robinWeakSol}
\underbrace{\int_{\partial \Omega} ik\psi v \, d\gamma - \int_\Omega \nabla \psi \nabla v \, d\Omega + \int_\Omega k^2 \psi v \, d\Omega} _\text{$a(\psi, v)$} = \underbrace{\int_\Omega gv \, d\Omega} _\text{F(v)}
\end{align}

When we insert for $\psi$ and move the sum outside, we get
\begin{align}\label{eq:robinMatrixForm}
\sum_{j=1}^n u_h^j \underbrace{\left( \int_{\partial \Omega} ik\phi_j \phi_i \, d\gamma - \int_\Omega \nabla \phi_j \nabla \phi_i \, d\Omega + \int_\Omega k^2 \phi_j \phi_i \, d\Omega \right)} _\text{$\tilde{{A_{ij}}}$} = \underbrace{\int_{\Omega} g\phi_i \, d\Omega} _\text{$b_i$}.
\end{align}

We get $n$ equations by choosing $v$ to be the basis functions, i.e. $\phi_i$ for $i = 1, \dots, n$. Then \eqref{eq:robinMatrixForm} is equivalent to $A u_h = b$, and we can solve for $u_h$.

Given $M$ edges on the boundary $\partial \Omega$, we can name the edges $e_1, \dots, e_M$. 

By fixing $i$ and $j$ and looking at the boundary term in $\tilde{A_{ij}}$ the value for the term will be
\begin{align}
\int_{\partial \Omega} ik(x)\phi_j \phi_i \, d\gamma = \sum_{k=1}^M \int_{e_k} ik(x)\phi_j \phi_i \, d\gamma\\
= \sum_{e \in \textrm{supp}(\phi_i) \cap \textrm{supp}(\phi_j)} \int_e \phi_j \phi_i \, d\gamma.
\end{align}





\end{document}


\begin{thebibliography}{}

\bibitem{Quarteroni}
Alfio Quarteroni, \emph{Numerical Models for Differential Problems}, 2nd Edition, 2014.

\bibitem{Project}
Ulrik Fjordholm, \emph{Programming project in TMA4220, part 2B:
Helmholtz' equation for electromagnetic waves} https://wiki.math.ntnu.no/$\_$media/tma4220/2015h/project2b.pdf, Nov 2015.

\bibitem{Helmholtz-Wave}
Tilo Arens, \emph{Helmholtz Equation - waveguides and scattering} https://people.maths.ox.ac.uk/trefethen/pdectb/helmholtz2.pdf, Nov 2015


\end{thebibliography}

\end{document}

