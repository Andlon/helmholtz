\documentclass[10pt,a4paper]{article}
\usepackage[latin1]{inputenc}
\usepackage{amsmath}
\usepackage{amsfonts}
\usepackage{amssymb}
\begin{document}


\section*{Robin-Type Boundary Conditions}
If we instead of the Dirichlet boundary conditions take a look at a Robin-type boundary condition, given by
\begin{align}\label{eq:robinType}
\frac{\partial \psi}{\partial n} - ik\psi = 0,
\end{align}
the idea is that we will obtain a solution that ensures that all waves will leave $\Omega$ and none will enter $\Omega$. \cite{Project}

Starting with the Helmholtz equation, we multiply by a test function $v$ and integrate over the domain $\Omega$. 

\begin{align}
\int_\Omega \Delta \psi v \, d\Omega + \int_\Omega k^2 \psi v \, d\Omega = \int_\Omega gv \, d\Omega
\end{align}

Doing integration by parts on the first term leads to
\begin{align}\label{eq:robinIntParts}
\int_{\partial \Omega} \frac{\partial \psi}{\partial n} v \, d\gamma - \int_\Omega \nabla \psi \nabla v \, d\Omega + \int_\Omega k^2 \psi v \, d\Omega = \int_\Omega gv \, d\Omega
\end{align}

Inserting \eqref{eq:robinType} into \eqref{eq:robinIntParts} we get our weak solution
\begin{align}\label{eq:robinWeakSol}
\underbrace{\int_{\partial \Omega} ik\psi v \, d\gamma - \int_\Omega \nabla \psi \nabla v \, d\Omega + \int_\Omega k^2 \psi v \, d\Omega} _\text{$a(\psi, v)$} = \underbrace{\int_\Omega gv \, d\Omega} _\text{F(v)}
\end{align}

When we insert for $\psi$ and move the sum outside, we get
\begin{align}\label{eq:robinMatrixForm}
\sum_{j=1}^n u_h^j \underbrace{\left( \int_{\partial \Omega} ik\phi_j \phi_i \, d\gamma - \int_\Omega \nabla \phi_j \nabla \phi_i \, d\Omega + \int_\Omega k^2 \phi_j \phi_i \, d\Omega \right)} _\text{$\tilde{{A_{ij}}}$} = \underbrace{\int_{\Omega} g\phi_i \, d\Omega} _\text{$b_i$}.
\end{align}

We get $n$ equations by choosing $v$ to be the basis functions, i.e. $\phi_i$ for $i = 1, \dots, n$. Then \eqref{eq:robinMatrixForm} is equivalent to $A u_h = b$, and we can solve for $u_h$.

Given $M$ edges on the boundary $\partial \Omega$, we can name the edges $e_1, \dots, e_M$. 

By fixing $i$ and $j$ and looking at the boundary term in $\tilde{A_{ij}}$ the value for the term will be
\begin{align}
\int_{\partial \Omega} ik(x)\phi_j \phi_i \, d\gamma = \sum_{k=1}^M \int_{e_k} ik(x)\phi_j \phi_i \, d\gamma\\
= \sum_{e \in \textrm{supp}(\phi_i) \cap \textrm{supp}(\phi_j)} \int_e \phi_j \phi_i \, d\gamma.
\end{align}





\end{document}