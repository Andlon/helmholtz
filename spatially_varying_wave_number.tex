\documentclass[10pt,a4paper]{article}
\usepackage[latin1]{inputenc}
\usepackage{amsmath}
\usepackage{amsfonts}
\usepackage{amssymb}
\begin{document}

\section*{Introducing Spatially Varying Wave Number}
By introducing a spatially varying wave number given by
\begin{align}\label{eq:spatialWaveNumber}
k(x) = \frac{\omega}{c(x)} = \frac{\omega}{\sqrt{\mu(x)\epsilon(x)}},
\end{align}
we are able to model a more realistic scenario for how the signal will behave in our domain. In our implementation we can create geometric shapes where all triangles enclosed in this shape will be of the same material. Each material has a different wave number calculated from the material's permeability $\mu$ and permittivity $\epsilon$, as given in \eqref{eq:spatialWaveNumber}.

In our final example, we have modeled a rectangular house with wooden walls, a circular pool with water, and a rectangular box with walls made of lead. The rest of the domain contains air. As the $k$-value will vary for each of the materials, we see that the signal behaves differently in regions of different materials.





\end{document}