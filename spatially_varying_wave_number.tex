\documentclass[10pt,a4paper]{article}
\usepackage[latin1]{inputenc}
\usepackage{amsmath}
\usepackage{amsfonts}
\usepackage{amssymb}
\begin{document}

\section*{Introducing Spatially Varying Wave Number}
By introducing different materials in our domain we are able to model a more realistic scenario. A material is given by its values for permeability $\mu$ and permittivity $\epsilon$, and each triangle has an assigned material. In our implementation we can create geometric shapes with different materials in the domain. A situation of spatially varying wave number will change how the signal behaves in a domain $\Omega$ of varying materials.



\end{document}