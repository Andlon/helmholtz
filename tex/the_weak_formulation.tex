\documentclass[10pt,a4paper]{article}
\usepackage[latin1]{inputenc}
\usepackage{amsmath}
\usepackage{amsfonts}
\usepackage{amssymb}
\begin{document}

\section*{The Weak Formulation}
We consider the following Dirichlet problem:
\begin{equation}
\begin{aligned}
\label{Helmholtz_Dir}
\Delta \psi + k^2 \psi &= g  \quad \text{in} \, \Omega \\
\psi &= 0 \quad \text{on} \, \partial \Omega.
\end{aligned}
\end{equation}
To derive a weak formulation the equation is multiplied with a suiting test function $v \in X $ and integrated over the domain $\Omega$. The left hand side gives:
\begin{align*}
\int_\Omega \nabla \psi v &+ k^2 \psi v \, \mathrm{d} \Omega \\
&\Downarrow \\
\int_{\partial \Omega} \frac{\partial \psi}{\partial n} v \, \mathrm{d} \Omega + \int_\Omega & k^2 \psi v - \nabla \psi \nabla v \, \mathrm{d} \Omega.
\end{align*}
At this point we can make a choice for the space of $v$. We choose $v$ to lie in the following space:
\begin{align*}
X = H^1_0(\Omega) = \left\{ v \in H^1(\Omega) : v = 0 \text{ on }   \partial \Omega \right\},
\end{align*}
where
\begin{align*}
H^1(\Omega) = \left\{ v: \Omega \rightarrow \mathbb{R}: v \in L^2(\Omega), v' \in L^2(\Omega) \right\}.
\end{align*}
Since $v$ disappears on the boundary we are left with the following expression:
\begin{align*}
\int_\Omega & k^2 \psi v - \nabla \psi \nabla v \, \mathrm{d} \Omega = \int_\Omega gv \, \mathrm{d} \Omega.
\end{align*}
Let now $a(u,v)$ and $F(v)$ be defined as the following functionals:
\begin{align*}
a(u,v) &= \int_\Omega  \nabla u \nabla v - k^2 u v  \, \mathrm{d} \Omega  \quad \text{and} \\
F(v) &= - \int_\Omega gv \, \mathrm{d} \Omega.
\end{align*}
This implies that $a(u,v) = F(v)$. The following problem is called the weak formulation of (\ref{Helmholtz_Dir}):
\begin{equation}
\begin{aligned}
\label{Helmholtz_weak}
\text{find} \, u \in H^1_0(\Omega): \quad a(u,v) = F(v) \quad \forall \, v \in H^1_0(\Omega).
\end{aligned}
\end{equation}
The functional $a(u,v)$ is continuous. This is seen by the following:
\begin{align*}
|a(u,v)| &= |\int_\Omega  \nabla u \nabla v - k^2 u v \, \mathrm{d} \Omega| \\
 		& \leq |\int_\Omega \nabla u \nabla v \, \mathrm{d} \Omega| + k^2 |\int_\Omega u v \, \mathrm{d} \Omega| \\
 		& \leq \vectornorm{\nabla u}_{L^2(\Omega)} \vectornorm{\nabla v}_{L^2(\Omega)} + k^2 \vectornorm{u}_{L^2(\Omega)} \vectornorm{v}_{L^2(\Omega)} \\
 		& \leq C \vectornorm{u}_{L^2(\Omega)} \vectornorm{v}_{L^2(\Omega)} + k^2 \vectornorm{u}_{L^2(\Omega)} \vectornorm{v}_{L^2(\Omega)} \\
 		& \leq C \vectornorm{u}_{H^1(\Omega)} \vectornorm{v}_{H^1(\Omega)} + k^2 \vectornorm{u}_{H^1(\Omega)} \vectornorm{v}_{H^1(\Omega)} \\
 		& = M \vectornorm{u}_{H^1(\Omega)} \vectornorm{v}_{H^1(\Omega)}.
\end{align*}
The functional $a(u,v)$ is also coercive under certain conditions, since
\begin{align*}
a(u,u) &= \int_\Omega  \nabla u \nabla u - k^2 u u \, \mathrm{d} \Omega \\
 		& = \int_\Omega  |\nabla u|^2 \, \mathrm{d} \Omega - \int_{\Omega} k^2 |u|^2 \, \mathrm{d} \Omega \\
 		& =  \vectornorm{\nabla u}_{L^2(\Omega)}^2 - k^2 \vectornorm{u}_{L^2(\Omega)}^2\\
 		& = \vectornorm{u}_{H^1(\Omega)}^2 - \vectornorm{u}_{L^2(\Omega)}^2 - k^2 \vectornorm{u}_{L^2(\Omega)}^2\\
 		& = \vectornorm{u}_{H^1(\Omega)}^2 - C \vectornorm{u}_{L^2(\Omega)}^2\\
 		& \geq \vectornorm{u}_{H^1(\Omega)}^2 - C \vectornorm{u}_{H^1(\Omega)}^2\\
 		& = \alpha \vectornorm{u}_{H^1(\Omega)}^2
\end{align*}
Lastly the functional is bilinear, since it is linear in each component. The bilinear functional $F(v)$ is continuous:
\begin{align*}
|F(v)| &= |\int_\Omega gv \, \mathrm{d} \Omega| \leq \vectornorm{g}_{L^2(\Omega)} \vectornorm{v}_{L^2(\Omega)} \\
       &= M \vectornorm{v}_{L^2(\Omega)},
\end{align*}
and from the Lax-Milgram theorem there exists a unique solution to (\ref{Helmholtz_weak}).

\end{document}