\documentclass[10pt,a4paper]{article}
\usepackage[latin1]{inputenc}
\usepackage{amsmath}
\usepackage{amsfonts}
\usepackage{amssymb}
\begin{document}

\section*{Introduction}
The Helmholtz equation,
\begin{equation}
\begin{aligned}
\label{Helmholtz}
\Delta \psi + k^2 \psi = g,
\end{aligned}
\end{equation}
is a time-independent form of the wave equations which often arise in physical problems involving the computation of electromagnetic radiation. In the modeling of electromagnetic waves $\psi$ represents the Fourier transform of either the electric field $E$ or the magnetic field $H$. $k$ is called the wavenumber, and from the derivation of the Helmholtz' equation from Maxwell's equations in vacuum it is seen that $k = \frac{2\pi f}{c}$. Here $f$ is the frequency of the electromagnetic waves and $c = \frac{1}{\sqrt{\mu_0 \epsilon_0}}$ is the speed of light, where $\mu_0$ and $\epsilon_0$ are the permeability and permittivity in vacuum.

\end{document}