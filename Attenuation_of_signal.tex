\documentclass[10pt,a4paper]{article}
\usepackage[latin1]{inputenc}
\usepackage{amsmath}
\usepackage{amsfonts}
\usepackage{amssymb}
\usepackage{mathtools}
\usepackage{bm}
\usepackage{standalone}
% Use \bm{x} for vectors/matrices in bold AND italic

\newcommand{\vectornorm}[1]{\left\|#1\right\|}

\begin{document}
To account for the attenuation of signal as it passes through materials it is useful to define a complex permittivity. To see why this will lead to a weaker signal (a smaller amplitude), we will look at the wavenumber $k(x)$ given by \eqref{eq:spatialWaveNumber} which will now be complex as $c(x)$ is complex. The wavenumber will now be varying as the signal moves through different materials. Let $k = a + ib$ and consider a solution to the wave equation which will have the general form $\phi = \psi_0 e^{i(kx- \omega t)}$. It is seen that 
\begin{align*}
\phi &= \psi_0 e^{i(kx- \omega t)} \\
 &\Downarrow \\
 \phi &= \psi_0 e^{-bx} e^{i(ax- \omega t)}, \\
\end{align*}
where the $e^{-bx}$ term of this equation will lead to an exponential decay of the signal for increasing $x$-values if $b>0$ ($b<0$ would make the signal grow exponentially, which has no physical interpretation).

The implementation of a complex wavenumber is straightforward since Matlab handles calculations with complex numbers the same way as for real numbers. Thus, no changes in the code had to be made to run the program using a complex permittivity. The resulting numerical solution is complex, and the amplitude in the wave equation is found by taking the absolute value of the complex solution.
\end{document}