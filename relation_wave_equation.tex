\documentclass[10pt,a4paper]{article}
\usepackage[latin1]{inputenc}
\usepackage{amsmath}
\usepackage{amsfonts}
\usepackage{amssymb}
\begin{document}


\section*{Relation to The Wave Equation}
As already stated, the Helmholtz equation for the modeling of electromagnetic waves can be derived from the Maxwell's equations in vacuum
\begin{align*}
\nabla \times E &= - \frac{\partial H}{\partial t} \\
\nabla \times H &= \mu_0 \epsilon _0 \frac{\partial E}{\partial t} \\
\nabla \cdot E &= 0 \\
\nabla \cdot H &= 0
\end{align*}

where $E: \Omega \times \mathbb{R}_+ \rightarrow \mathbb{R}^3$ is the electric field and $H: \Omega \times \mathbb{R}_+ \rightarrow \mathbb{R}^3$ is the magnetic field. It can be shown (\cite{Project}) that the equations can be decoupled into the wave equation for each component of the electric and magnetic field,
\begin{equation}
\begin{aligned}
\label{Wave}
\frac{\partial^2 \phi}{\partial t^2} - c^2 \Delta \phi = 0, 
\end{aligned}
\end{equation}
where $\phi$ is a spatial component for $E$ or $H$. To arrive at the Helmholtz equation we take the Fourier transform of $\phi$. From \cite{Helmholtz-Wave} we have that if $u(x,t) = \psi (x)e^{-i \omega t}$ satisfies (\ref{Wave}), then $\psi (x)$ is a solution to (\ref{Helmholtz}). This means that by numerically solving the Helmholtz equation, we are modeling the amplitude of the wave for one of the spatial components of the two fields $E$ and $H$.

\end{document}